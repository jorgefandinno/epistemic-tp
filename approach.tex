\section{Approach}\label{sec:approach}

Let~$\Pi$ be a positive program.
%
We define the operators
\begin{align}
    \TKP{\wv} \quad&=\quad \SM[\Pi^\wv]
    \\
    \TKPn{0}{\wv} \quad&=\quad \wv
    \\
    \TKPn{i+1}{\wv} \quad&=\quad \TKP{\TKPn{i}{\wv}}
\end{align}
For epistemic interpretations~$\wv$ and~$\wv'$, we write~$\wv \sqsubseteq \wv'$ if~$\wv' \models \K\fF$ for all formulas~$\fF$ such that~$\wv \models \K\fF$.

\begin{conjecture}
    \label{conj:tp.monotone}
    Operator~$\TKPo$ is monotone and, thus, it has a least fixpoint and there is some~$n \geq 0$ such that~$\TKPn{n}{[\emptyset]} = \TKPn{n+1}{[\emptyset]} $.
\end{conjecture}

By~$\TKPf$ we denote the least fixpoint of operator~$\TKPo$.

\begin{conjecture}
    \label{conj:tp.g94}
    $\TKPf$ is a G94-world view of~$\Pi$ for any positive program~$\Pi$.
\end{conjecture}

\begin{conjecture}
    \label{conj:tp.founded}
    $\TKPf$ is founded for any positive program~$\Pi$.
\end{conjecture}


We define the reduct wrt. to an epistemic interpretation~$\wv$, written~$\Pi^{\underline{\wv}}$, as the set of programs
\begin{gather}
    \Pi^{\underline{\wv}} \quad = \quad \{ \ \Pi^I \mid I \in \wv \ \}
\end{gather}
We now extend the $\TKPo$ operator to sets of programs. Given a set of programs~$\Omega$, we define
\begin{align}
    \TK{\Omega}{\wv} \quad&=\quad \bigcup_{\Pi \in \Omega} \SM[\Pi^\wv]
\end{align}

\begin{conjecture}
    \label{conj:tp.monotone}
    Operator~$\TKo{\Omega}$ is monotone and, thus, it has a least fixpoint and there is some~$n \geq 0$ such that~$\TKn{\Omega}{n}{[\emptyset]} = \TKn{\Omega}{n+1}{[\emptyset]} $.
\end{conjecture}

By~$\TKPf$ we denote the least fixpoint of operator~$\TKPo$.

\begin{conjecture}
    \label{conj:tp.g94}
    $\TKf{\Omega}$ is a G94-world view of~$\Pi$ for any eazy program~$\Pi$.
\end{conjecture}

\begin{conjecture}
    \label{conj:tp.founded}
    $\TKf{\Omega}$ is founded for any eazy program~$\Pi$.
\end{conjecture}

We define now a partial reduct that only evaluates the epistemic literals that are true in the current world view.
%
The \emph{positive reduct} of a theory~$\Gamma$ 
with respect to an epistemic interpretation~$\wv$, written~$\Gamma^{\uparrow\wv}$, is obtained by replacing by $\top$ each maximal subformula~$\fF$ of the form~$\K \fG$ such that~$\wv \models \fF$.


For a positive program~$\Pi$ and a set of constraints~$\Delta$, we define the operators
\begin{align}
    \hat{K}_{\Pi,\Delta}(\wv) \quad&=\quad \{ \ I \in \wv' \mid \wv' =  \hat{K}(\Pi^{\uparrow\wv}) \text{ and }\tuple{\wv',I} \models \Delta \ \}
    \\
    \hat{K}_{\Pi,\Delta}^0(\wv) \quad&=\quad \wv
    \\
    \hat{K}_{\Pi,\Delta}^{i+1}(\wv) \quad&=\quad \hat{K}_{\Pi,\Delta}(\hat{K}_{\Pi,\Delta}^i(\wv))
\end{align}

\begin{conjecture}
    \label{conj:tp2.monotone}
    Operator~$\hat{K}_\Pi$ is monotone and, thus, it has a least fixpoint and there is some~$n \geq 0$ such that~$\hat{K}_\Pi^n([\emptyset]) = \hat{K}_\Pi^{n+1}([\emptyset])$.
\end{conjecture}

By~$\tilde{K}(\Pi,\Delta)$ we denote the least fixpoint of operator~$\hat{K}_{\Pi,\Delta}$.

\begin{conjecture}
    \label{conj:tp2.g94}
    For any positive program~$\Pi$ and set of constraints~$\Delta$, if~$\hat{K}(\Pi,\Delta)$ is non\nobreakdash-empty, then it is a G94-world view of~$\Pi \cup \Delta$.
\end{conjecture}

Examples:
\begin{align}
    a & \leftarrow \neg \K \neg a
\end{align}
is translted to
\begin{align}
    a & \leftarrow \neg b
    \\
    b & \leftarrow \K c
    \\
    c & \leftarrow \neg a
\end{align}
and then to
\begin{align}
    a & \leftarrow \overline{b}
    \\
    b & \leftarrow \K c
    \\
    c & \leftarrow \overline{a}
    \\
    \{ \overline{a} \} & \leftarrow
    \\
    & \leftarrow a \wedge \overline{a}
    \\
    & \leftarrow \neg a \wedge \neg \overline{a}
    \\
    \{ \overline{b} \} & \leftarrow
    \\
    & \leftarrow b \wedge \overline{b}
    \\
    & \leftarrow \neg b \wedge \neg \overline{b}
\end{align}
Let us consider some candidates to be world views. First~$[ \{ \overline{a}, b, c \} ]$.
%
The reduct is
\begin{align}
    a & \leftarrow \overline{b}
    \\
    b & \leftarrow \K c
    \\
    c & \leftarrow \overline{a}
    \\
    \overline{a} & \leftarrow
    \\
    & \leftarrow a \wedge \overline{a}
    \\
    & \leftarrow \neg a \wedge \neg \overline{a}
    \\
    & \leftarrow b \wedge \overline{b}
    \\
    & \leftarrow \neg b \wedge \neg \overline{b}
\end{align}
and this program has this as a construtive world view.
%
Second~$[ \{ a, \overline{b} \} ]$.
%
The reduct is
\begin{align}
    a & \leftarrow \overline{b}
    \\
    b & \leftarrow \K c
    \\
    c & \leftarrow \overline{a}
    \\
    \overline{b} & \leftarrow
    \\
    & \leftarrow a \wedge \overline{a}
    \\
    & \leftarrow \neg a \wedge \neg \overline{a}
    \\
    & \leftarrow b \wedge \overline{b}
    \\
    & \leftarrow \neg b \wedge \neg \overline{b}
\end{align}
and this program has this as a construtive world view.
%
Finally, let us consider~$[ \{ \overline{a}, b, c \},\, \{ a, \overline{b} \} ]$.
%
Then, we get the set~$\Omega$ containing the above two programs.
%
We have
\begin{align}
    \TKPnn{0} \quad & =\quad   [\emptyset]
    \\
    \TKPnn{n} \quad & =\quad   [\{\overline{a}, c \},\, \{ a, \overline{b} \}]
\end{align}
with~$n \geq 1$.
%
Hence, not a world view.

Another example.
\begin{align}
    a \vee b & \leftarrow 
    \\
    a & \leftarrow \K a
    \\
    c & \leftarrow \K a
    \\
    & \leftarrow \neg c
\end{align}
Let us consider some candidates to be world views. First~$[ \{ a,c \} ]$.
%
The reduct is the result of removing the last rule and this is its world view.

Another example.
\begin{align}
    a \vee b & \leftarrow 
    \\
    a & \leftarrow \K a
    \\
    c & \leftarrow \K a
    \\
    & \leftarrow \neg c
    \\
    & \leftarrow \neg a
\end{align}
DOES NOT WORK!